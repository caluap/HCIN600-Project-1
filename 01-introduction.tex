\section{Introduction}

As its harsh realities spread %dizzyingly 
over everyday life, climate change has become a dominant theme in societal discourse. While discussions about the extent of the damage already done, our shared future prospects, and possible mitigation strategies are still a highly politically charged topic, even traditionally denialist blocs have started acknowledging that \emph{something} must be done \cite{teirstein_2021}.  

In computing, the discussion about how the field is related to and could act on climate change is not new but has become more prominent in recent years. Considering scholarly work, for instance, in the \textsc{acm} Digital Library there were \oldstylenums{295} climate change-related papers published between \oldstylenums{1991} to \oldstylenums{2009}, with another \oldstylenums{2,095} then published from \oldstylenums{2010} to \oldstylenums{2020} \cite{ferreiraClimateChangeCommunication2021}. As a complex, multi-variate issue, different researchers have adopted varied approaches about what role computing should play. \citet{easterbrook2010climate} argues that the software community has to ``step up to the plate,'' as other fields have done, and calls for a discipline-wide effort on work on software to ``support the science of understanding climate change[,] to support the global collective decision making[, and] to reduce the carbon footprint of modern technology.''

This fits what \citet{silberman2010precarious} call \emph{adaptation-oriented pre-apocalyptic computing}, which assumes a future apocalypse\footnote{The term, which the authors acknowledge as perhaps too \emph{inflammatory} for scholarly use, ``denotes an event in the intersection of the spaces of events denoted by the terms \emph{collapse} and \emph{disaster}.''} is likely, but attempts to ``develop tools for users to negotiate it successfully,'' making it ``less apocalyptic [with] materials and social relations available at design time (pre-apocalypse) but likely to be unavailable at use time.'' An alternative approach, \emph{adaptation-oriented post-apocalyptic computing} works within material constraints as if the apocalypse had already happened, thus anticipating and easing a transition towards future conditions --- which, interestingly, makes this approach share methods with the already-present realities of ``disaster informatics, community informatics, \textsc{ict\oldstylenums{4}d}, \textsc{hci\oldstylenums{4}d}, humanitarian logistics, sociotechnical action research and `post normal science''' --- for many underdeveloped regions around the world, the ``ingredients of apocalyptic computings and other related practices [constituting] deindustrial techne'' are are already a given.

In that vein, \citet{nardiComputingLimits2018} challenge computing's assumed premise that ``exponential growth of computing capacity and an ever-expanding infrastructure for computing will continue into the future,'' positing that, as world-limits make themselves felt, an alternative, and even optimistic, approach would be to strive for a ``transformative change to a system more like steady-state economy,'' i.e. one not predicated on continuous economic (and resource consumption) growth. 
\citet{wong2009prepare} goes further: in denouncing how designers ``often confuse needs with desire,'' they claim that they ``will need to recognize the pervasiveness and insidiousness of denial in materialistic populations,'' and that perhaps a reality shock (brought upon a collapse of much of what makes present-day computing possible) will allow sustainable interaction design to ``shift its focus from persuading to sustaining the human race.''

All of this begets the question: how is the new generation of computing professionals readying themselves to act within this scenario that, while bleak, will inevitably shape their future careers? While climate change affects the field as a whole, new and old professionals alike, we chose to focus on computing students since their mental models about computing's role in the Anthropocene are supposedly still being shaped by their on-going education. This, we hypothesize, might give a good indication of where the field is heading. As such, this study attempts to answer the following research questions:

\begin{enumerate}
    \item If they feel that climate change will impact how their field of expertise works?
    \item If they do, what do computing students think their field of expertise will look like during the effects of climate change?
\end{enumerate}