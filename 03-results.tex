\section{Results}

\emph{(Participants' names coded for anonymity, and their responses edited for clarity and brevity.)}

    \subsection{How are individuals affected by Climate Change?}
    
    \begin{quoting}
        \textit{``It's affecting everybody, it affects humans, animals, at the end of the day, the tech field are still people who get adversely affected by climate, right? So honestly, I don't know if there is a demarcation between tech people and non-tech people when it comes to climate change, it just affects everybody regardless.''} (\pr{1})
    \end{quoting}

    From the interviews, it is observed that the all the participants believe that climate change is urgent and needs to be planned for. \pr{1} mentions that although the effects of climate change will be seen on the tech field, it is not isolated to the tech field and should be considered as a global problem.
    
    \begin{quoting}
        \textit{``Also, I mean, I was listening to this podcast, I think it is called `How to change the world'... it's a climate podcast, and one of the episodes was about the rising… It was about Miami, and how the sea level continues to rise, and how they are just building, you know, they continue to elevate their buildings and, no one really talks enough about how this is an issue, you know, we can't keep putting our buildings on stilts.''} (\pr{2})
    \end{quoting}
    
    \begin{quoting}
        \textit{``Particularly sensitive to climate change I would think of the energy field. Fossil fuels are a big part and climate change. I would also add real estate, because if the ice caps melt and the sea levels rise as you're seeing in some places already, those beach side properties will be underwater.''} (\pr{4})
    \end{quoting}

    \begin{quoting}
        \textit{``Actually, I think in city it is also urgent or sensitive, because as more and more people are moving from rural areas to cities, this will increase CO\textsubscript{2} emissions, which is one of the major causes for the global warming.''} (\pr{5})
    \end{quoting}
    
    \begin{quoting}
        \textit{``I think the lower economic section gets affected by it more, but the upper economic section can reduce the impact of climate change for themselves. For example, if you are a very rich farmer, you can get irrigation or better seeds, but if you are a poor farmer and the rains are not on time because of climate change, then you are in a bad position.''} (\pr{1})
    \end{quoting}

    All participants agree that there are certain areas that are more sensitive and will be affected adversely by climate change. \pr{2} and \pr{4} talk about the melting icecaps and rising water levels in the oceans and how this is problem for cities built on the coast. \citet{hitz2004estimating} illustrate the rising sea levels will cause loss of land, larger damages from storms, saltwater intrusion and increased cost in coastal defenses. \pr{5} feels that due to the migration of citizens from rural to urban areas there will be an increase in emission of CO\textsubscript{2}, contributing to global warming --- which, it is worth noting, is not in line with recent research on the topic \cite{castells2020density}. \pr{1} feels that rather than only looking at the effects from a global standpoint, we must also look at them for groups from lower economic backgrounds, who will be less prepared for the effects of Climate Change.
    
    \begin{quoting}
        \textit{``I mean, I feel like I don't personally fear for myself getting, you know… I am not worried about [whether] where I live down there is going to catch on fire''} (\pr{2})
    \end{quoting}

    While most participants talked about how the drastic change in climate would affect the population of earth, they did not mention how these changes would personally affect them. \pr{2} did mention that although they were concerned about the effects of climate change, they felt that it would not personally change their plans.
    \subsection{How individuals can affect Climate Change}
    
    Our interview participants all acknowledged that climate change is indeed an urgent issue and can be addressed through both monitoring individually daily usage of technology and raising awareness of power consumption of certain computing fields. However, there are challenges to adopt a more environmentally friendly lifestyle.
    
    \subsubsection{Individual Behavior Changes}
    
    Several participants express the intention to incorporate personal behavior changes to combat climate change. It's worth noting that electric cars are often mentioned as something they would considering purchasing:
    
    \begin{quoting}
        \textit{``I'm becoming more conscious about the power that tech uses. For example, I used to leave my PlayStation on rest mode, but now I power it off. I know it's not much, but if a million people started doing it, that's a big deal.''} (\pr{1}) % as it only uses about 4 watts per hour
    \end{quoting}
    
    \begin{quoting}
        \textit{``I may change my private car to energy-saving and environment-friendly in the future. Electric cars are a good choice.''} (\pr{3})
    \end{quoting}
    
    % Concerns over transferring to electric vehicles were also brought up by one participant.

    \begin{quoting}
        \textit{``I've read suggestions of using your current appliances till the end of their lifespan, and to then recycle them, avoiding byproducts that would harm the environment. For instance, I plan on buying an eclectic car or hybrid in the future, but I'll drive my current car until it is no longer operable.''} (\pr{4})
    \end{quoting}
    
    One participant mentioned that structural change should be more emphasized in term of impact on climate change:
    
    \begin{quoting}
        \textit{``I do not think that the everyday use of technology will change much, but those big companies that consume a lot of plastics [will have to change].''} (\pr{5})
    \end{quoting}
    
    \subsubsection{Challenges of changing lifestyles}
    
    While most participants stated that they would love to adapt an environmentally friendly and energy-saving lifestyle, they also realized that such change can be unpleasant and hard to execute. Affordability is the major concerns for some of our participants. %Partly due to the demographic distribution in our research, 
    
    \begin{quoting}
        \textit{``I would say that I couldn't afford an electric car right now, but that would be an ideal plan.''} (\pr{2})
    \end{quoting}
    
    \begin{quoting}
        \textit{``It again comes down to the trade-off, because the greener the tech you want right now, the more expensive it gets. Very weird situation but unfortunately that's what it is. If it's within budget, or even if it's reasonably above budget, then I would go for a greener option, but yeah at the end of the day it comes down to economics.''} (\pr{1})
    \end{quoting}
    
    The tension between a comfortable lifestyle versus an inconvenient and expensive but environmentally friendly one is present in many participants' answers. %what cause people's hesitation of the change
    
    \begin{quoting}
        \textit{``We're so used to having everything accessible within days or hours. I can order something from across the country and it'll be here in two days, and I can do that through an app, through the click of my fingers, and I think that's something that… you know, as someone who's in the field of user experience, and making things easier for users, I think, sure, we want to make things easy for users, and accessible, but I feel like we also don't want to neglect being environmentally conscious, and we don't want to make things too easy that we are lazy and neglectful''} (\pr{2})
    \end{quoting}

    
\subsection{How are computing fields can affect Climate Change}

    % This results section is spread out according to two progressive core questions.
    
    % \subsubsection{Interview Question: Do you feel your specific field presents unique opportunities in terms of alleviating the effects of climate change? And how impactful do you think they could be?}
    
    Many participants believe their own computing field of expertise can both worsen and help alleviate climate change's effects: %, especially advantage in large-scale computing.
        
    % \begin{quoting}
    %     \textit{
    %     "[The] problem can be visualized and predicted, based on existing receipts generated images and analysis." (\pr{3})
    %     }
    % \end{quoting}
    
    \begin{quoting}
        \textit{
        ``As we know that all [of the effects of] climate change can be recorded in the database, if we want to get more useful information, we cannot ignore the powerful tools of the AI.''       }(\pr{5})
    \end{quoting}
    
    \begin{quoting}
        \textit{
        ``There is a lot of room where we can optimize (looking to optimize power efficiency in data centers). There are a lot of industries where ML and AI could help.''
        }  (\pr{1})
    \end{quoting}
    
    \begin{quoting}
        \textit{
        ``I think that [games] would let people be more conscious about how they affect the climate.''
        }(\pr{4})
    \end{quoting}
    
    One participant reported their ambiguous and confused attitude when considering the field of user experience:
    
    \begin{quoting}
        \textit{
        ``We want to make things easy and accessible for users, but I feel like we also don't want to neglect being environmentally conscious, and we don't want to make things too easy that we are lazy and neglectful. I definitely will order things from Amazon, and I'm like, `that's probably not the best thing to do. Why don't I just go outside of my apartment, walk down the street and buy things locally?'\thinspace''  % Which are things that have become more.
        }(\pr{2})
    \end{quoting}
    
    \begin{quoting}
        \textit{
        ``When I was an undergrad [in architecture], the topic of climate change was very much in my everyday language, and since taking the \textsc{hci} path, it kind of lost it a bit...''        }(\pr{2})
    \end{quoting}
    
    Some participants also reported the negative effects that computing has on climate change. Most of these concerns were related to large-scale energy consumption:
    
    \begin{quoting}
        \textit{
        ``[Cloud services] need to consume a tremendous amount of energy and, to maintain operating speed, additional power is needed to dissipate heat.''
        } (\pr{3})
    \end{quoting}
    
    \begin{quoting}
        \textit{
        ``For some research doing large-scale machine learning, they will use a lot of electricity[, and] their long-term goals are not aligned with a green climate trend.'' (\pr{5})
        }
    \end{quoting}
    
    % Further, discussed how impactful computing could be for alleviating the effects of climate change. 
    
    Some participants mentioned that even though computing brings light to the climate change issue, it is just a piece of the puzzle in terms of solving climate change. For example, one participant considered that, in most cases, technology could be just a complementary tool. Another participant reported that what current computing can do is to make small and minor improvements. %And depending on the scale, it will make a difference.
    
    Meanwhile, some participants kept uncertain attitudes. For example, one participant mentioned that social media impacted people to a big degree, but they were not sure about the level of influence of social media could have specifically on climate change:
    
    \begin{quoting}
        \textit{
        ``[In the] computer field, our advantages are very strong social communication and influence. But depending on past experience, good media tend to spread [virally]. However, we won't know how people will behave after transmission.'' 
        }(\pr{3})
    \end{quoting}
    
