\section{Discussion}

\subsection{Unsystematic risks}

All participants were unanimous in stating that they felt climate change as a real and urgent issue. In their understanding of the scope of the risks posed by it, however, its effects were generally perceived as if of an unsystematic nature, in line with what \citeauthor{silberman2010precarious} call \emph{emergencies}, i.e., ``events that have impacts on social units, which mobilize responses to these impacts,'' but which do not usually \emph{exceed} societies' capabilities for response.

This is seen both in how they imagine themselves potentially affected by climate change and how they feel the computing field will (if at all) change in response to it. P2, for instance, knows that Albuquerque, New Mexico (where they plan to move soon), is suffering from out-of-control wildfires and on-going issue with major droughts. Yet, in their speech, these seem as if inconveniences that are not necessarily deal-breakers.

While there were hints of how parts of computing are interconnected --- e.g. P4 saying that ``if you were working with hardware [and you could not get the resources because] they would be shutting down the plant'' ---, the more general perception is that there are certain \emph{localized} risks that could cause disruptions, but not a \emph{collapse}. The societal status quo is seen as inherently stable.

In this sense, participants seem to call for a loose \emph{adaptation-oriented pre-apocalyptic computing} within a very limited scope. They propose specific changes, both for the field and their personal lives, but these are discontinuous and, more especially, non-structural: on a personal level, many mentioned wanting to change an \textsc{ice} car for an electric one, but there was little discussion of car-dependency embedded in city design; in computing, there was mention of the use of green components in computer manufacturing, but consumerism itself was only briefly mentioned.

P2 hinted at this contradiction between how there are limits to what non-structural changes are capable of achieving when they pointed that convenience, as it is embedded in common everyday apps and services, is contradictory with environmentalism, but the contradiction seemed unsolvable since ``companies should not make `bad apps.'\thinspace''

\subsection{Limits to action}

A second theme we perceived in participants' answers was that they seemed to see themselves as if helpless to initiate any major changes, be it societal or in their own fields of expertise. Not only did they generally avoid discussing the need for structural changes, what changes they did suggest or see as forthcoming are all implicitly to be initiated by unknowable forces, over which the interviewees seem to have no say.

This, again, seems to be tied to a notion of climate change-related risks as if geographically and chronologically far off, and to the belief that whatever society emerges from it will be a superficially improved but structurally similar version of ours.

Even if one assumes, like \citeauthor{easterbrook2010climate}, an \emph{adaptation-oriented pre-apocalyptic computing} scenario, mitigation is necessarily a costly endeavor and one which needs careful coordination and intense effort. In this case, this passivity is worrying, as it seems to indicate that reflections about climate change are still superficial.

It also worrying when one considers that, if \emph{collapses} do occur, current students might be ill-equipped to adapt to new circumstances. If, as states \citeauthor{wong2009prepare}, ``our future lacks some of the technology opulence we have grown to expect, there are still plenty of opportunities for sustainable interaction design to have an impact,'' it would seem important to explore some of these opportunities at our present \emph{pre-apocalyptic} context.