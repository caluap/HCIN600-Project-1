\section{Discussion}

\subsection{Unsystematic risks}

All participants were unanimous in stating that they felt climate change as a real and urgent issue. In their understanding of the scope of the risks posed by it, however, its effects were generally perceived as if of an unsystematic nature. This is in line with what \citeauthor{silberman2010precarious} call \emph{emergencies}, i.e., ``events that have impacts on social units, which mobilize responses to these impacts,'' but which do not usually \emph{exceed} societies' capabilities for response.

This is seen both in terms of how they perceive they can be personally affected by climate change and how they feel the computing field will (if at all) change in response to it. P2, for instance, knows that Albuquerque, New Mexico (where she plans to move soon), is suffering with out-of-control wildfires, and also that there has been an on-going issue with major droughts. Yet, in her speech these seem as if inconveniences, but not necessarily deal-breakers.

While there were hints of how parts of computing are interconnected --- e.g. P4 saying that ``if you were working with hardware [and you could not get the resources because] they would be shutting down the plant'' ---, the more general perception is that there are certain \emph{localized} risks that could cause disruptions, but not a \emph{collapse}, as if societal status quo was inherently stable.

In this sense, participants seen to call for a loose \emph{adaptation-oriented pre-apocalyptic computing}, but with a very limited scope. They propose specific changes, both for the field and their personal lives, but these are discontinuous and, more specially, non-structural: on a personal level, many mentioned wanting to change an \textsc{ice} car for an electric one, but there was little discussion of car-dependency embedded in city design; in computing, there was mention of the use of green components in computer manufacturing, but consumerism itself was just briefly mentioned.

P2 hinted at this contradiction between how there are limits to what non-structural changes are capable of achieving when they pointed that convenience, as it is embedded in common everyday apps and services, is contradictory with environmentalism, which seems unsolvable since ``companies should not make `bad apps.'\thinspace''

\subsection{A monolithic field}

\subsection{Limitations}


