
\section{Conclusions and Future work}
As a way to potentially help the debate about how the computing field can evolve to better deal with climate change, and also as a query into how its next generation of practitioners imagine the field will change in the coming years, our research interviewed computing students from different educational levels. 

We asked them about their perceptions about the interactions between the field and climate change. In response to our first research question (\emph{if they feel that climate change will impact how their field of expertise works?}), the general answer seems to be that yes, the field will change, even if the extent to which this is perceived as true seems to be quite limited, and it is not clear to our participants who should initiate these changes. 

This feeds into our second research question (\emph{what do computing students think their field of expertise will look like during the effects of climate change?}), for which our answers pointed to a field that is only superficially different than it currently is at its present state.

These answers are worrying for two reasons. First, they indicate that, even if climate change is named as an urgent matter, computing seems to have a passive attitude towards it, both in terms of protecting itself against probable disastrous future scenarios and in terms of helping society to avert them. 

Second, the answers also show that these students themselves might be receiving an education that aims an each day more unlikely future professional landscape, while at the same time ignoring steps they could be taking now to prepare themselves for this \emph{post-apocalyptic computing}, to use \citeauthor{silberman2010precarious}'s term.

Given these bleak findings, it is urgent that climate change adaptation is given a greater role in computing courses. This could be through a greater emphasis in topics like those mentioned by \citeauthor{silberman2010precarious}, \citeauthor{easterbrook2010climate}, among others. 

\subsection{Future work} 

One gap left by our research is: do computing students actually think what is currently being done as enough to deal with climate change? Are they optimistic, or, if pessimistic, what should change? Also, it would be worth querying the reasons behind the current passivity we have found.

To maybe gain a more in-depth understanding of how computing students could react to more structural changes to their field, further research could present some of the alternative scenarios discussed here and probe how students think they would fare in each of them, and if their opinions of the current status quo would change.


% The future research directions could be:
% \begin{itemize}
%     \item Research over the perceptions held by people of other fields, such as biology, agriculture, politics, etc;
%     \item One major finding in this paper is that people tend to lack a more profound understanding towards a structural paradigm to address the climate change. One research direction could then be to interview people to explore their views on people's impact on a more grand level;
%     \item Another aspect worth exploring is the opinions of people on how satisfactory they are with the current status quo of efforts put into mitigating climate change, or the thoughts on how their own field has contributed to help with the climate change;
%     \item One research question could be to explore the attitudes towards the success of overcoming the climate change. Do people hold an optimistic or a pessimistic view towards the future of climate change?
% \end{itemize}


