
\section{Conclusions and Future work}
As a way to potentially help the debate about how the computing field can evolve to better deal with climate change, and also as a query into how its next generation of practitioners imagine the field will change in the coming years, our research interviewed computing students from different educational levels. 

We asked them about their perceptions about the interactions between the field and climate change. In response to our first research question (\emph{if they feel that climate change will impact how their field of expertise works?}), the general answer seems to be that yes, the field will change, even if the extent to which this is perceived as true seems to be quite limited, and it is not clear to our participants who should initiate these changes. 

This feeds into our second research question (\emph{what do computing students think their field of expertise will look like during the effects of climate change?}), for which our answers pointed to a field that is only superficially different than its present state.

These answers are worrying for two reasons. First, they indicate that, even if climate change is named as an urgent matter, computing seems to have a passive attitude towards it, both in terms of protecting itself against probable disastrous future scenarios and in terms of helping society to avert them. 

Second, the answers also show that these students themselves might be receiving an education that aims an each day more unlikely future professional landscape, while at the same time ignoring steps they could be taking now to prepare themselves for this \emph{post-apocalyptic computing}, to use \citeauthor{silberman2010precarious}'s term.

Given these bleak findings, it is urgent that climate change adaptation is given a greater role in computing courses. This could be through a greater emphasis in topics like those mentioned by \citeauthor{silberman2010precarious}, \citeauthor{easterbrook2010climate}, among others. 

\subsection{Future work}


\subsection{Limitations of this study}
For one, time restrictions limited our number of interviewees to only five. While we were able to obtain a varied set of responses, sub-fields of computing were underrepresented, when at all, i.e., we only had one participant from \textsc{hci}, one from game development, etc. Considering how \cite{easterbrook2010climate} proposes sub-field related responses, a broader and more representative sample could deepen the discussion of more specific educational interventions, for instance.

The small sample also meant that we did not achieve data saturation, so while there were common themes between participants' responses, there was also a lot of variety among them. Given our decision to design a short interview guide, more participants could increase confidence in our thematic analysis.


