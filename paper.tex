%%
%% This is file `sample-authordraft.tex',
%% generated with the docstrip utility.
%%
%% The original source files were:
%%
%% samples.dtx  (with options: `authordraft')
%% 
%% IMPORTANT NOTICE:
%% 
%% For the copyright see the source file.
%% 
%% Any modified versions of this file must be renamed
%% with new filenames distinct from sample-authordraft.tex.
%% 
%% For distribution of the original source see the terms
%% for copying and modification in the file samples.dtx.
%% 
%% This generated file may be distributed as long as the
%% original source files, as listed above, are part of the
%% same distribution. (The sources need not necessarily be
%% in the same archive or directory.)
%%
%% The first command in your LaTeX source must be the \documentclass command.
% \documentclass[sigconf,authordraft]{acmart}
\documentclass[sigconf]{acmart}

\settopmatter{printacmref=false} % Removes citation information below abstract
\renewcommand\footnotetextcopyrightpermission[1]{} % removes footnote with conference information in first column
\pagestyle{plain} % removes running headers


%% NOTE that a single column version may required for 
%% submission and peer review. This can be done by changing
%% the \doucmentclass[...]{acmart} in this template to 
%% \documentclass[manuscript,screen]{acmart}
%% 
%% To ensure 100% compatibility, please check the white list of
%% approved LaTeX packages to be used with the Master Article Template at
%% https://www.acm.org/publications/taps/whitelist-of-latex-packages 
%% before creating your document. The white list page provides 
%% information on how to submit additional LaTeX packages for 
%% review and adoption.
%% Fonts used in the template cannot be substituted; margin 
%% adjustments are not allowed.

%%
%% \BibTeX command to typeset BibTeX logo in the docs
\AtBeginDocument{%
  \providecommand\BibTeX{{%
    \normalfont B\kern-0.5em{\scshape i\kern-0.25em b}\kern-0.8em\TeX}}}


\setcopyright{none}
\acmConference[]{}{}{}

\begin{document}

\title{The Name of the Title is Hope}

\author{\textsc{hcin}600 group 4}
\affiliation{%
  \institution{Rochester Institute of Technology}
  \city{Rochester}
  \state{New York}
  \country{USA}
}

% \author{Narayanan Asuri Krishnan}
% \affiliation{%
%   \institution{Rochester Institute of Technology}
%   \city{Rochester}
%   \state{New York}
%   \country{USA}
% }
% \email{nk1581@rit.edu}

% \author{Xin Miao Lin}
% \affiliation{%
%   \institution{Rochester Institute of Technology}
%   \city{Rochester}
%   \state{New York}
%   \country{USA}
% }
% \email{xl3439@rit.edu}

% \author{Caluã de Lacerda Pataca}
% \email{cd4610@rit.edu}
% % \authornote{All authors contributed equally to this research.}
% \affiliation{%
%   \institution{Rochester Institute of Technology}
%   \city{Rochester}
%   \state{New York}
%   \country{USA}
% }

% \author{Yuting Shao}
% \affiliation{%
%   \institution{Rochester Institute of Technology}
%   \city{Rochester}
%   \state{New York}
%   \country{USA}
% }
% \email{ys2884@rit.edu }

% \author{Xiaoyin Xi}
% \affiliation{%
%   \institution{Rochester Institute of Technology}
%   \city{Rochester}
%   \state{New York}
%   \country{USA}
% }
% \email{xx4455@rit.edu}

%%
%% By default, the full list of authors will be used in the page
%% headers. Often, this list is too long, and will overlap
%% other information printed in the page headers. This command allows
%% the author to define a more concise list
%% of authors' names for this purpose.
\renewcommand{\shortauthors}{Group 4}

%%
%% The abstract is a short summary of the work to be presented in the
%% article.
\begin{abstract}
\end{abstract}

%%
%% The code below is generated by the tool at http://dl.acm.org/ccs.cfm.
%% Please copy and paste the code instead of the example below.
%%
% \begin{CCSXML}
% <ccs2012>
%  <concept>
%   <concept_id>10010520.10010553.10010562</concept_id>
%   <concept_desc>Computer systems organization~Embedded systems</concept_desc>
%   <concept_significance>500</concept_significance>
%  </concept>
%  <concept>
%   <concept_id>10010520.10010575.10010755</concept_id>
%   <concept_desc>Computer systems organization~Redundancy</concept_desc>
%   <concept_significance>300</concept_significance>
%  </concept>
%  <concept>
%   <concept_id>10010520.10010553.10010554</concept_id>
%   <concept_desc>Computer systems organization~Robotics</concept_desc>
%   <concept_significance>100</concept_significance>
%  </concept>
%  <concept>
%   <concept_id>10003033.10003083.10003095</concept_id>
%   <concept_desc>Networks~Network reliability</concept_desc>
%   <concept_significance>100</concept_significance>
%  </concept>
% </ccs2012>
% \end{CCSXML}

% \ccsdesc[500]{Computer systems organization~Embedded systems}
% \ccsdesc[300]{Computer systems organization~Redundancy}
% \ccsdesc{Computer systems organization~Robotics}
% \ccsdesc[100]{Networks~Network reliability}

%%
%% Keywords. The author(s) should pick words that accurately describe
%% the work being presented. Separate the keywords with commas.
% \keywords{climate change, collapse computing}

%% A "teaser" image appears between the author and affiliation
%% information and the body of the document, and typically spans the
%% page.
% \begin{teaserfigure}
%   \includegraphics[width=\textwidth]{sampleteaser}
%   \caption{Seattle Mariners at Spring Training, 2010.}
%   \Description{Enjoying the baseball game from the third-base
%   seats. Ichiro Suzuki preparing to bat.}
%   \label{fig:teaser}
% \end{teaserfigure}

%%
%% This command processes the author and affiliation and title
%% information and builds the first part of the formatted document.
\maketitle

\section{Introduction}

As its harsh realities spread %dizzyingly 
over everyday life, climate change has become a dominant theme in societal discourse. While discussions of the extent of the damage already done, our shared future prospects, and possible mitigation strategies are still highly politically charged, even traditionally denialist blocs have started acknowledging that \emph{something} must be done. \cite{teirstein_2021} 

In computing, the discussion about how the field is related to and could potentially act on climate change is not new but has become more prominent in recent years. In the \textsc{acm} Digital Library, for instance, 295 climate change-related papers were published from 1991 to 2009, with 2,095 then published from 2010 to 2020. \cite{ferreiraClimateChangeCommunication2021} As a complex, multi-variate issue, different researchers adopt varied approaches about what role computing could, or should, play. \citet{easterbrook2010climate} argues that the software community has to ``step up to the plate,'' as other fields have done, and calls for a discipline-wide effort on work on three areas: 
    \begin{quote}
        Software to support the science of understanding climate change; software to support the global collective decision making; and software to reduce the carbon footprint of modern technology.
    \end{quote}

This fits what \citet{silberman2010precarious} call \emph{adaptation-oriented pre-apocalyptic computing}, which assumes a future apocalypse\footnote{The term, which the authors acknowledge as perhaps too \emph{inflammatory} for scholarly use, ``denotes an event in the intersection of the spaces of events denoted by the terms \emph{collapse} and \emph{disaster}.''} is likely, but attempts to ``develop tools for users to negotiate it successfully,'' making it ``less apocalyptic [with] materials and social relations available at design time (pre-apocalypse) but likely to be unavailable at use time.'' An alternative approach, \emph{adaptation-oriented post-apocalyptic computing} works within material constraints as if the apocalypse had already happened, thus anticipating and easing a transition towards future conditions --- which, interestingly, makes this approach share methods with the already-present realities of ``disaster informatics, community informatics, \textsc{ict4d}, \textsc{hci4d}, humanitarian logistics, sociotechnical action research and `post normal science.'''

... falar do \citet{wong2009prepare}


% \begin{quote}
%     The idea that exponential growth of computing capacity and an ever-expanding infrastructure for computing will continue into the future is usually taken for granted. \cite{nardiComputingLimits2018}
% \end{quote}


% \begin{quote}
%     Sustainable HCI (...) [and c]limate change communication (...) [are] strongly interrelated since increasing the quality of information about the topic is key to changing people's mindsets. \cite{ferreiraClimateChangeCommunication2021}
% \end{quote}


% RQ1: Do computing students feel that climate change will impact how their field of expertise works? RQ2: If they do, what do computing students think their field of expertise will look like during the effects of climate change?

\section{Methods}

\citep{braun2006using}

\section{Results}
\section{Discussion}
\section{Conclusions and Future work}



\bibliographystyle{ACM-Reference-Format}
\bibliography{refs}

\end{document}
\endinput
%%
%% End of file `sample-authordraft.tex'.
